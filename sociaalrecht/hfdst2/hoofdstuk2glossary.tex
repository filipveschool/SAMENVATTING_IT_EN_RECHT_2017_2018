\begin{comment}
%==> dit is het woord waarnaar verwezen wordt in de tekst
\newglossaryentry{}{
%==> dit is hetgene dat wordt getoond in de glossaries lijst
name={},
%==> dit is de beschrijving van de glossary
description={}
}

%==> dit is het woord waarnaar verwezen wordt in de tekst
\newglossaryentry{}{
%==> dit is hetgene dat wordt getoond in de glossaries lijst
name={},
%==> dit is de beschrijving van de glossary
description={}
}


\end{comment}


%\newglossaryentry{accesscontrollist}{
%name={Access Control List },
%description={ }
%}

%==> dit is het woord waarnaar verwezen wordt in de tekst
\newglossaryentry{proefbeding}{
%==> dit is hetgene dat wordt getoond in de glossaries lijst
name={Proefbeding},
%==> dit is de beschrijving van de glossary
description={Is een beperkte periode vanaf de aanvang van de tewerkstelling waarin beide partijen op een soepele wijze een einde kunnen maken aan de arbeidovereenkomst.}
}

%==> dit is het woord waarnaar verwezen wordt in de tekst
\newglossaryentry{concurrentiebeding}{
%==> dit is hetgene dat wordt getoond in de glossaries lijst
name={concurrentiebeding},
%==> dit is de beschrijving van de glossary
description={Is een clausule in de arbeidsovereenkomst waarbij de werknemer de verbintenis aangaat om geen soortgelijke activiteiten uit te oefenen wanneer hij de onderneming verlaat, hetzij door zelf een onderneming uit te baten, hetzij door in dienst te treden bij een concurrerende werkgever.}
}

%==> dit is het woord waarnaar verwezen wordt in de tekst
\newglossaryentry{scholingsbeding}{
%==> dit is hetgene dat wordt getoond in de glossaries lijst
name={Scholingsbeding},
%==> dit is de beschrijving van de glossary
description={Is een verbintenis die de werknemer aangaat om een bepaalde tijd in dienst te blijven bij zijn werkgever omdat hij op diens kosten een opleiding mag volgen.}
}
