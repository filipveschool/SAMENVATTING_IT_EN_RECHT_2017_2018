\chapter{Schorsing van de uitvoering van de arbeidsovereenkomst}
\label{hoofdstuk:2}

\section{Het begrip schorsing}

\section{Overmacht}

\subsection{Begrip}

\subsection{Gevolgen}

\subsection{Recht op werkloosheidsuitkeringen}

\section{Het gewaarborgd dagloon}

\subsection{Met vertraging of niet op het werk aankomen}

\subsection{De arbeid niet kunnen beginnen of voortzetten}

\section{Arbeidsongeschiktheid}

\subsection{Schorsing}

\subsection{Verplichtingen van de werknemer}

\subsubsection{Meldingsplicht}

\subsubsection{Geneeskundig getuigschrift}

\subsection{Controle door de werkgever}

\subsection{Betwistingen van medische aard}

\subsection{Gewaarborgd loon}

\subsubsection{Algemeen}

\subsubsection{Gewaarborgd dagloon (art. 27 WAO) en gewaarborgd loon bij arbeidsongeschiktheid}

\subsubsection{Uitsluiting}

\subsubsection{Recht van verhaal tegen de aansprakelijke derde}

\subsection{Het bedrag van het gewaarborgd loon}

\subsubsection{De regeling voor werklieden (WAO en cao nr. 12\textit{bis}}

\paragraph{Carenzdag}

\paragraph{Periode en bedrag}

\paragraph{Anciënniteitsvoorwaarde}

\paragraph{Hervallen}

\paragraph{Gewaarborgd loon bij arbeidsongeval of beroepsziekte}

\subsubsection{De regeling voor bedienden}

\paragraph{Algemeen stelsel}

\paragraph{Bedienden die de regeling voor werklieden volgen}

\paragraph{Gewaarborgd loon bij arbeidsongeval of beroepsziekte}

\subsubsection{Overzicht van het gewaarborgd loon bij ziekte}

\section{Kort verzuim (of ook: klein verlet)}

\subsection{Algemene principes}

\subsection{Lijst van de gebeurtenissen}

\subsection{Geboorteverlof}

\subsection{Adoptieverlof}



%%% Local Variables: 
%%% mode: latex
%%% TeX-master: "masterproef"
%%% End: 
