\chapter{Einde van de arbeidsovereenkomst}
\label{hoofdstuk:2}

\section{Algemene (civielrechtelijke) wijzen van beëindiging}

\section{Wederzijdse toestemming}

\section{Overmacht}

\section{Ontbindend beding}

\section{Afloop van de termijn of voltooiing van het afgesproken werk}

\section{Overlijden van een van de partijen}

\section{Arbeidsrechtelijke wijzen van beëindiging: 'ontslag' en 'opzeg'}

\subsection{Algemeen}

\subsection{De opzeg}

\subsubsection{Algemeen}

\subsubsection{Wijze van kennisgeving van de opzeg}

\paragraph{Opzeg uitgaande van de werknemer}

\paragraph{Opzeg uitgaande van de werkgever}

\subsubsection{Inhoud}

\subsubsection{Nietigheid van de opzeg}

\subsubsection{Opzegtermijn}

\paragraph{Begrip - Wat voorafging aan de Wet Eenheidsstatuut}

\paragraph{Aanvang van de termijn}

\paragraph{Belang van de anciënniteit voor de duur van de opzegtermijn}

\paragraph{Duur van de opzegtermijn: algemeen}

\paragraph{Opzegtermijn: algemene regels}

\paragraph{Uitzonderingen op de opzegtermijnen}

\paragraph{Geen afwijkingen in een cao gesloten in een paritair (sub)comité}

\paragraph{Opzegtermijn in geval van pensioen}

\paragraph{Schorsing van de opzegtermijn}

\subparagraphsection{Algemeen}

\subparagraphsection{Schorsingsoorzaken en invloed op de opzegtermijn}

\paragraph{Recht op afwezigheid tijdens de opzegtermijn: sollicitatieverlof}

\paragraph{De opzegvergoeding}

\subparagraphsection{Algemeen}

\subparagraphsection{Loon als berekeningsbasis voor de opzegvergoeding}

\section{Ontslag wegens dringende reden}

\subsection{Algemeen}

\subsubsection{Begrip dringende reden}

\subsubsection{Voorwaarden}

\paragraph{"Een ernstige tekortkoming"}

\paragraph{"Die elke professionele samenwerking onmogelijk maakt"}

\paragraph{"De samenwerking moet onmiddelijk en definitief onmogelijk gemaakt zijn"}

\subsection{Termijnen}

\subsubsection{Termijn om tot ontslag over te gaan}

\subsubsection{Termijn en kennisgeving van de motieven}

\subsubsection{Bewijs}

\subsubsection{Vergoedingsregeling}

\subsubsection{Voorbeelden uit de rechtspraak}

\section{Motivering van het ontslag en kennelijk onredelijk ontslag}

\subsection{Algemeen}

\subsection{Motiveringsplicht}

\subsection{Kennelijk onredelijk ontslag}

\subsubsection{Inhoud}

\subsubsection{Uitzonderingen}

\subsection{Willekeurig ontslag van arbeiders}

\section{Onregelmatige beëindiging of verbreking van de arbeidsovereenkomst}

\subsection{Algemeen}

\subsection{Opzegvergoeding}

\section{Bescherming tegen ontslag}

\section{Rechtsmisbruik bij ontslag}

\subsection{Algemeen}

\subsection{Rechtsmisbruik bij ontslag}

\section{Het ontslag van een arbeidsongeschikte werknemer}




%%% Local Variables: 
%%% mode: latex
%%% TeX-master: "masterproef"
%%% End: 
